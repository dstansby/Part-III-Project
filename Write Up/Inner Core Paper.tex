\documentclass[11pt,a4paper]{article}
\usepackage{fullpage}	% Bigger pages
\usepackage{cite} 		% Include citations
\usepackage{hyperref} 	% Hyperlinks everywhere!
\usepackage{natbib}		% Include bibliography
\usepackage{graphicx}	% Include pictures
\usepackage{amsmath}	% Use \eqref

\begin{document}

\title{Constraining the Hemispherical Structure in the Hidden Layer At the Top of the Earth's Inner Core}
\author{Candidate Number: \\  \\ Supervisor: Prof. Keith Priestley}
\maketitle

\begin{abstract}
Since its discovery in 1936, the Earth's inner core has been well documented by both body wave and normal mode studies. However, one area where properties are not yet well measured is the top of the inner core. The upper region of the inner core is of particular interest as it is thought that as the outer core freezes onto the inner core the variable environment at this boundary is encoded in the properties of the frozen material. 
\end{abstract}

\tableofcontents

\newpage
\section{Introduction}

\section{Theoretical Background}
This project centres around using seismic body wave analysis in order to investigate the velocity structure of the upper inner core. These are elastic waves, caused by earthquakes, that travel through the interior of the Earth. Under the assumptions of a continuous, linearly elastic medium, infinitesimal strains and a locally uniform medium one can derive the elastic wave equation\footnote{This section is essentially summarised from \cite{Shearer2009}}
\begin{equation}
	\rho \frac{\partial^{2} \vec{u}}{\partial t^{2}} = \left ( \lambda + \mu \right ) \nabla \left ( \nabla \cdot \vec{u} \right ) + \mu \nabla^{2} \vec{u}
	\label{eq:Wave Equation}
\end{equation}
where u is the displacement vector, $\rho$ the density of the medium and $\lambda$ and $\mu$  Lam\'{e} parameters of the medium. A general displacement vector can be decomposed into a irrotational scalar and solenoidal vector potentials such that
\begin{equation}
	u = \nabla \phi + \nabla \times \vec{\psi}
	\label{eq:Displacement}
\end{equation}

Substituting \eqref{eq:Displacement} in to \eqref{eq:Wave Equation} yields two independent wave equations, one for $\phi$ and one for $\vec{\psi}$, which describe P-waves and S-waves respectively. P-waves are compressional with motion occurring parallel to the wave vector, whereas S-waves are transverse with motion occurring perpendicular to the wave vector.

RAY THEORY HERE

Because the outer core is liquid with $\mu \approx 0$ and thus does not transmit S-waves, it is P-waves that are used to sample the inner core.

\section{Papers}
\begin{itemize}
	\item \cite{Nissen-Meyer2014} - Describes the AxiSEM waveform modelling software. Discusses the need for full waveform modelling, and the computational constraints that AxiSEM overcomes.

	\item \cite{Waszek2013a} - Calculate attenuation properties, after taking into account velocity structure from \cite{Waszek2011a}.
\begin{itemize}
	\item Large enough earthquakes to provide visible signal.
	\item Deep enough to prevent surface reflection interference.
	\item Filter from 0.7 Hz to 2 Hz.
\end{itemize}

\end{itemize}
% Bibliography
\bibliographystyle{yahapj}
\bibliography{/Users/dstansby/Physics/Papers/library}

\end{document}