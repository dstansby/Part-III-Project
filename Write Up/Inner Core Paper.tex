\documentclass[11pt,a4paper]{article}
\usepackage{fullpage}	% Bigger pages
\usepackage{cite} 		% Include citations
\usepackage[breaklinks,colorlinks,urlcolor=blue,citecolor=blue,linkcolor=blue]{hyperref} 	% Hyperlinks everywhere!
\usepackage{natbib}		% Include bibliography
\usepackage{graphicx}	% Include pictures

\begin{document}

\title{Project Plan - Constraining the Hemispherical Structure in the Hidden Layer At the Top of the Earth's Inner Core}
\author{David Stansby \\ ds598@gmail.com \\  \\ Supervisor: Prof. Keith Priestley \\ kfp10@cam.ac.uk }
\maketitle

\begin{abstract}
Since its discovery in 1936, the Earth's inner core has been well documented by both body wave and normal mode studies. However, one area where properties are not yet well measured is the top of the inner core. The upper region of the inner core is of particular interest as it is thought that as the outer core freezes onto the inner core the variable environment at this boundary is encoded in the properties of the frozen material. 
\end{abstract}

\section{Introduction}

\section{Papers}
\begin{itemize}
	\item \cite{Nissen-Meyer2014} - Describes the AxiSEM waveform modelling software. Discusses the need for full waveform modelling, and the computational constraints that AxiSEM overcomes.

	\item \cite{Waszek2013a} - Calculate attenuation properties, after taking into account velocity structure from \cite{Waszek2011a}.
\begin{itemize}
	\item Large enough earthquakes to provide visible signal.
	\item Deep enough to prevent surface reflection interference.
	\item Filter from 0.7 Hz to 2 Hz.
\end{itemize}

\end{itemize}
% Bibliography
\bibliographystyle{yahapj}
\bibliography{/Users/dstansby/Physics/Papers/library}

\end{document}