\documentclass[10pt,a4paper,twocolumn]{article}
\usepackage{fullpage}


\begin{document}

\title{Constraining the Hemispherical Structure in the Hidden Layer At the Top of the Earth's Inner Core}
\author{David Stansby \\ ds598@gmail.com }
\maketitle

\begin{abstract}
\end{abstract}

\section{Scientific Motivation}
The properties of the inner core are measured using the arrival times of two difference seismic waves generated by earthquakes, PKiKP and PKIKP. Figure PUT FIGURE HERE shows the ray paths for each of these waves. The reference PKiKP wave reflects off the Inner Core Boundary (ICB), whereas the PKIKP wave travels within the Inner Core. Any differences between these two waves is therefore a measure if the inner core structure.

Figure PUT FIGURE HERE shows an example seismogram with PKiKP and PKIKP arrivals marked on. As the maximum depth that the PKIKP wave travels beneath the ICB is reduced, the differential travel time reduces causing the individual wave packets to overlap, limiting our ability to distinguish them. It is for this reason that the structure at the top of the inner core is poorly measured at present.

There are several methods that could be used in order to resolve this problem.	

Also of particular interest is the anisotropy boundary present in the inner core.

\section{Project Goals}
The project has two primary goals:
\begin{enumerate}
	\item How does varying the sharpness of the inner core hemisphere boundaries affect the arrival times and waveforms of PKiKP and PKIKP waves?
	\item Are any methods available that allow the arrival of PKiKP and PKIKP waves to be distinguished when the two waveforms start to overlap, and if so what is the minimum differential arrival time that can be resolved using these methods?
\end{enumerate}
\section{Project Methodology}
The project will be primarily computation based, and carried out from my personal laptop. The Bullard Laboratory's supercomputer will also be employed in order to create synthetic seismograms.
\section{Project Plan}
The project breaks down into a number of sections, each of which depends on all previous tasks being completed.

\begin{itemize}
	\item Carry out a comprehensive research review, identifying the current state of research in this area.
	\item Identify a high quality data set that can be used.
\end{itemize}

\end{document}